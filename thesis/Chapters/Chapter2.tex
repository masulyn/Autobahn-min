\chapter{Background} 

\label{Chapter2} % \ref{ChapterX}

%----------------------------------------------------------------------------------------
%	SECTION 1
%----------------------------------------------------------------------------------------

\section{GHC Profiling and Cost Centers}

Intuitively, \Ao{}'s genetic algorithm generates too many bangs in Pat
and Chris' program
because it conducts a purely random search for locations to insert bangs.
To focus the search on bangs that are likely to impact performance, \At{} makes use of the information provided by GHC's profiling system.
This system allows users to better understand which locations
in a program consume the most resources. The system adds annotations
to the program and generates a report detailing the amount of time,
memory allocations, or heap usage that is attributable to each location.
To generate these profiles, users simply run their program on
representative input after re-compiling it with options to request profiling.
Users can choose to generate either a time and allocation or a heap
profile, as well as the method by which the profiling system adds
annotations. Users can manually specify
annotations or invoke the profiler with  the \texttt{-prof -fprof-auto} option, which
automatically adds an annotation around every binding in the program
that is not marked \texttt{INLINE}. In the generated profile, each
annotation gives rise to a \textit{cost center} that indicates how
much time or memory were attributable to the given program point as a
percentage of the whole program's resource utilization.

%----------------------------------------------------------------------------------------
%	SECTION 2
%----------------------------------------------------------------------------------------

\section{Genes and Chromosomes}

Cost center profiling provides guidance for the otherwise random
search that \Ao{} performs using a genetic algorithm. In the
algorithm, any program source location where a bang may be added is
represented as a \textit{gene} that can be turned \textit{on},
corresponding to the presence of a bang, or \textit{off},
corresponding to its absence. Since it doesn't make sense to put two
bangs in one program location, there are a fixed number of possible
bang locations in a given program. A \textit{chromosome} comprises all
of the genes within a file. We represent a chromosome as a
fixed-length bit vector, in which each bit indicates the presence or
absence of a bang at the corresponding location. Since a program is a
collection of source files, it is represented as a collection of bit
vectors, or chromosomes.

%----------------------------------------------------------------------------------------
%   SECTION 3 
%----------------------------------------------------------------------------------------

\section{\Ao{}'s Genetic Algorithm}

\Ao{}'s genetic algorithm evaluates and manipulates randomly generated
chromosomes. It repeatedly generates new chromosomes before measuring
their performance using a fitness function. We call a chromosome that
either significantly slows down program performance or causes non
termination an \textit{\unfit{}} chromosome. If the fitness function
determines that a chromosome is \unfit{}, the chromosome is
immediately discarded. If the fitness function determines that a
chromosome behaved well, the chromosome is deemed \textit{\fit{}} and
kept for future rounds of generation.

For each round of chromosome generation, \Ao{} introduces randomness
by performing either a mutation or a crossover. A mutation flips a
gene in the chromosome whenever a randomly chosen floating point
number exceeds the \textit{mutateProb} threshold. A crossover combines
two chromosomes by randomly picking half of the genes from each parent
chromosome. For either of these random operations, \Ao{} uses a unique
number generator each time to guarantee randomness.

%----------------------------------------------------------------------------------------
%   SECTION 4 
%----------------------------------------------------------------------------------------

\section{Optimization Coverage}

\Ao{} uses program source files as the unit of granularity for
the set of program locations to consider when inferring bangs. By
default, \Ao{} optimizes all source code files in the program's
directory. In other words, \Ao{} by default represents all files in the directory using a list of chromosomes in which each chromosome corresponds to one file. However, Pat and Chris can specify a different set of files by
manually adding or removing file paths in the \Ao{} configuration.
We call the set of files considered in a given run of \Ao{}
its \textit{optimization coverage}.
\Ao{} does not infer bangs for external libraries
that are imported by the program, but Pat and Chris can manually add local
copies of the source code for such libraries to include them in the optimization process.

%----------------------------------------------------------------------------------------
%   SECTION 5 
%----------------------------------------------------------------------------------------

\section{Representative Input}

To run \Ao{}, Pat and Chris need to provide input on which to run their
program. The input should be short enough for \Ao{} to finish
execution in a reasonable amount of time while still be long enough
for it to measure noticeable time improvements if there are
any. The input should also be representative of the data that Pat and Chris
intend to supply to the finished program so the system optimizes the
program for that kind of data. It is this ability to optimize for
relevant input and not all input that gives bangs, whether added by
Autobahn or by the user, the ability to outperform GHC's optimizer.
