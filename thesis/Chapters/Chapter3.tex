\chapter{\At{}} 

\label{Chapter3} % \ref{ChapterX}

%----------------------------------------------------------------------------------------
%	SECTION 1
%----------------------------------------------------------------------------------------

\section{Why Too Many Bangs Are Generated}

The first step towards reducing the number of inferred bangs is to identify
categories of bangs and to hypothesize why each such category exists.
To that end, we identify three categories of bangs:
\begin{enumerate}
  \item{a \textit{\dangerous{}} bang is one that can significantly
degrade program performance, including causing non-termination}
  \item{a \textit{\useless{}} bang neither improves nor degrades
performance but is undesirable because programmers must waste
valuable time reasoning about its safety}
  \item{a \textit{\useful{}} bang improves program performance}
\end{enumerate}

An \unfit{} chromosome performs poorly as a whole, but it can contain
a mixture of \dangerous, \useless{}, and \useful{} bangs (By
definition, it must contain at least one \dangerous{} bang).
When interatively measuring performances of candidate chromosomes,
\Ao{} handles \unfit{} chromosomes by discarding them entirely.
It does not attempt to identify the \dangerous{} bangs within the chromosome. 
Individuallly removing such bangs from consideration, rather than simply discarding 
the chromosome they belong in, would
be useful because otherwise they can reappear in later generations as
the result of a random mutation.

Fit chromosomes face a similar issue. \Ao{} handles \fit{} chromosomes
by preserving the entire chromosome, without separating the \useful{}
bangs from the \useless{} ones. This lack of discrimination is
problematic for two reasons. First, we might lose \useful{} bangs in
future generations because we cannot guarantee that they will be
preserved by the mutation and crossover operations of the genetic
algorithm. Second, \useless{} bangs can survive if they are grouped
with \useful{} ones. The accumulation of such \useless{} bangs
can dramatically increase the number of inferred bangs, leaving the
user with a substantial reasoning task. As program source code
increases in size, the number of \useless{} bang positions also grows,
increasing the likelihood the genetic algorithm will infer and
preserve \useless{} bangs that happen to live on a chromosome
with \useful{} ones.

%----------------------------------------------------------------------------------------
%	SECTION 2
%----------------------------------------------------------------------------------------

\section{The Solution}

\subsection{Identifying Bang Categories}

Precisely identifying which bangs are \useless{} or \useful{} is undecidable, but
we can use GHC profiling to make an educated guess. 
Cost centers not only break down a chromosome into
smaller portions by source code bindings, but their associated costs
also imply how likely a bang placement will affect program
performance. 
Intuitively, a bang that appears in a cost center that uses many resources, which we
call a \textit{\hotspot}, is more likely to be \useful{} because a bang-induced
change in performance at the \hotspot{} will likely significantly affect overall
runtime as well. On the other hand, a bang in a cost center using few resources, 
which we call a \textit{\coldspot}, is likely to be \useless{} because executing
code at a \coldspot{} occupies an insignificant portion of the overall program
runtime to begin with. Leveraging that intuition, \At{} preserves bangs that appear
in \hotspots{} while eliminating those in \coldspots.

We introduce the \hotspotcost{} parameter to determine
which cost centers to consider hot:  a cost center consuming more
than \hotspotcost{} resources is considered hot, while
one consuming fewer is cold.
Currently, we set \hotspotcost{}
to 6\% of the overall program runtime, although that threshold can
be adjusted through user specification. 
As the threshold increases, \At{} preserves fewer bangs
at the risk of greater degradations in program performance.

\subsection{\At{}'s Phases}

\At{} uses GHC profiling information to reduce the
number of generated bangs in two different ways, corresponding to two
different phases in the optimization pipeline, which is shown in \figref{fig:pipeline}. 
The first phase, which we call the \Preopt{} phase, eliminates
any chromosome corresponding to a program source code file that
contains only \coldspots{}. \At{} invokes \Ao{} only on chromosomes
corresponding to files that contain at least one \hotspot{}. The \preopt{}
phase is beneficial for two reasons. Firstly, removing
chromosomes that are unlikely to contain \useful{} bangs reduces the
size of the search space with little chance of inadvertently
eliminating good sets of annotations. The smaller search space is
likely to make \Ao{} more effective in improving performance as the 
optimizer will have a chance to explore more possible combinations
of annotations, therefore producing higher quality bang candidates. 
Secondly, the optimizer will not be able to
infer bangs for any of the genes in the eliminated chromosomes.  
Since such bangs are very likely to have been \useless{} as they
are only located in \coldspots{}, this step is very likely to 
reduce the number of inferred \useless{} bangs.

The second phase, which we call the \Postopt{} phase, uses profiling
information to categorize \useless{} or \useful{} bangs that occur within \hotspots{} on
chromosomes that \Ao{} determines to be fit. In a world with infinite resources,
we would test all combinations of such bangs and select the best one
because the effect of combining bangs is hard to predict.
Unfortunately, the exponential search space is too large to explore
exhaustively. Instead, we individually turn off one such bang at a
time, measure the resulting performance, and keep only those that
significantly impact performance.  The number of such bangs is
sufficiently small that we can test each in turn because the number of
bangs in \hotspots{} in a program is relatively small.

In the rest of this section, we explain \At{}'s optimization pipeline
in more detail.

\begin{figure}
\tikzset{%
  >={Latex[width=4mm,length=2mm]},
            base/.style = {rectangle, rounded corners, draw=black,
                           minimum width=4cm, minimum height=1cm,
                           text centered, font=\sffamily},
  activityStarts/.style = {base, fill=blue!30},
       startstop/.style = {base, text width=6cm, fill=red!30},
    activityRuns/.style = {base, fill=green!30},
         process/.style = {base, text width=6cm, fill=orange!15,
                           font=\ttfamily},
}

\begin{tikzpicture}[node distance=3cm,
    every node/.style={fill=white, font=\sffamily}, align=center]
  \node (preo)             [activityStarts]              {\Preopt{}};
  \node (user)     [process, left of=preo, xshift=-5cm]          {Original Program, Representative Input, Autobahn Configuration};
  \node (autobahn)      [activityStarts, below of=preo, yshift=-1cm]
                                                      {\Ao{}};
  \node (end)      [startstop, left of=autobahn, xshift=-5cm, yshift=1cm]
                                                       {If program is unsuitable for optimization: Alert user and end process};

  \node (posto) [activityStarts, below of=autobahn, yshift=-0.6cm]
                                                    {\Postopt{}};
  \node (endmin)      [startstop, below of=end, yshift=0.2cm]
                                                       {If negligible optimization improvement: Alert user and end process};
   \node (result)     [process, left of=posto, xshift=-5cm, yshift=-1cm]          {Optimized and minimized program};

  \draw[->]             (preo) -- node[text width=6cm]
                    {Remove \coldspots{}, suggest external libraries, check if program is unsuitable for optimization}(autobahn);
  \draw[->]     (user) -- (preo);
  \draw[->]      (autobahn) -- node[text width=5cm]
                    {Find winning chromosome using genetic algorithms}(posto);
   \draw[->]      (preo) -- (end);
   \draw[->]      (posto) -- (endmin);
  \draw[->]      (posto) -- node[text width=4.3cm, xshift=1cm, yshift=-0.8cm] {Individually test each bang in winning chromosome}
                                   (result);
  \end{tikzpicture}
\scaption{\At{} optimization pipeline.}
\label{fig:pipeline}
\end{figure}

%----------------------------------------------------------------------------------------
%   SECTION 3 
%----------------------------------------------------------------------------------------

\section{\Preopt{} Profiling}

Just as manually reasoning about bang placement can be difficult,
so too reasoning about which files to optimize can be challenging.
\At{} makes this process easier by leveraging information
in GHC profiles.
To that end, the \preopt{} phase begins by generating a GHC time
and allocation profile for the unoptimized program by running it on
user-provided representative input.
\At{} then sets the optimization coverage
for \Ao{} to be source files that contain at least one \hotspot{}.
In addition,
\At{} identifies library files that contain \hotspots{} and
suggests to users that they add local copies so the library source
files may be optimized as well.
\Ao{} then optimizes the program as usual, except using a more
targeted set of files/chromosomes.
Note that this phase can both expand the search space, if a user
chooses to add the identified
library files that contain hotspots, as well as reduce it, by
identifying program source files with no \hotspots{}.

\Preopt{} profiling offers three important benefits.
First, it can greatly reduce the number of bangs \Ao{} suggests by
limiting the search space to those program files that have a chance of
significantly impacting performance. This focusing reduces the
possibility of generating \useless{} or \dangerous{} bangs by
eliminating them from consideration and
increases the chances of generating \useful{} ones by allowing more of
the relevant search space to be explored within a given time budget.
Second, if a \hotspot{} is located in an external library file,
the \preopt{} phase can identify the relevant files so they can be
included in the optimization process.
Third, it identifies programs that are potentially unsuitable for
\Ao{} optimization. If a program contains a large number of cost
centers that all contribute minimally to program runtime, there may
not be any way to substantially improve program performance by adding
bang annotations. If the \preopt{} phase concludes that
a program only contains \coldspots{}, it will alert the user and
abort, saving the time and effort of running the remaining phases,
which are unlikely to identify significant performance improvements.

It is worth noting that although the \preopt{} phase can reduce the
size of the search space, it does not reduce \Ao{}'s runtime.
\Ao{} will always search until exhausting its time budget.
The reduced search space does allow \Ao{} to explore the space
more thoroughly, statistically speaking, potentially allowing it to
find better results within the fixed time budget.

%----------------------------------------------------------------------------------------
%   SECTION 4 
%----------------------------------------------------------------------------------------

\section{\Postopt{} Bang Elimination}

After \Ao{} explores the search space and suggests a winning set of
chromosomes, \At{} uses GHC profiling information to eliminate any
bangs on those chromosomes that don't significantly contribute to
program performance. To that end,
the \postopt{} phase decides that any gene that is off in the winning
chromosomes should not have a bang in the final recommendation.
It removes these genes from further consideration.
The \postopt{} phase then maps each remaining gene, which must be
turned on, to its corresponding cost center.  It turns off all genes
that do not fall within \hotspots{}, deciding not to recommend bangs
for the corresponding locations on the theory that such bangs are
likely to be \useless{}.  It then removes these genes from further
consideration.

The remaining genes are the interesting ones in that they both
contain a bang and fall within a \hotspot{}. These genes require
further testing because they may still be \useless{}, failing to
improve program performance even though they fall within a \hotspot{}.
\cut{If they are \useless{}, they should be discarded. }
There is usually a sufficiently small number of remaining genes that
are both turned on and within a \hotspot{} that it is possible to
measure their impact one by one. Specifically,
the \postopt{} phase tests each such gene in turn, turning
it off while keeping all the remaining bangs on.  It
then runs the resulting program and compares its performance to that
of the winning chromosomes as determined by \Ao{}.
If the absence of the bang slows down the program by the value of the \absim{}
threshold parameter, the \postopt{} phase deems the bang useful and
decides to keep it.
Otherwise, the bang is deemed \useless{} and is
discarded. The \absim{} threshold is adjustable; we set it to
6\%. The \postopt{} phase repeats this process for every bang
that is both in the winning chromosomes and in a \hotspot{}.
The minimization result is the combination of all the surviving
bangs.

It is possible that there are no surviving bangs at the end of testing, and that occurs when
there are no \hotspot{} bangs remaining after all the \coldspot{} bangs are eliminated.
Another possibility is that the combination of \hotspot{} bangs prior to testing slows down
the program runtime to equal to or more than the original runtime, in which case the \preopt{}
phase will remove all bangs and return the original program.

%----------------------------------------------------------------------------------------
%   SECTION 5 
%----------------------------------------------------------------------------------------

\section{Representative Input}

Just as with \Ao{}, the quality of the representative input impacts the quality of
\At{}'s performance because different inputs may
generate wildly different results in GHC profiles. Therefore, users
must carefully choose input that well represents typical types of data the program
might receive.
