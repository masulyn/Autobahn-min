\chapter{Implementation} 

\label{Chapter4} % \ref{ChapterX}

%----------------------------------------------------------------------------------------
%	SECTION 1
%----------------------------------------------------------------------------------------

\section{Program Architecture}

\At{} wraps \Ao{} with the  \preopt{} and \postopt{} phases as shown
in \figref{fig:pipeline}.
Prior to running the genetic algorithm, the \preopt{} phase invokes
GHC's profiler on the unoptimized version of the user's program with
the \texttt{-prof -fprof-auto} option to auto-generate cost centers
around every non-inlined program binding.  It uses the supplied
representative data as input to generate the time and
allocation profile.  This baseline profile serves as the single point of
reference for the rest of \At{} to refer to for \hotspot{}
information throughout the entire optimization pipeline.
Depending on the where the \hotspots{} fall according to the profiler,
the program's optimization coverage will either be
automatically reduced or manually expanded by the user.

Then \At{} reuses the existing implementation
of \Ao{}~\cite{autobahn-wang} to identify a winning set of
chromosomes.
\Ao{} uses the \textit{haskell-src-exts}~\cite{langexts} parser to parse
source files and to identify genes.
It then applies a genetic algorithm,
implemented using the \texttt{GA} Haskell library~\cite{genetic}
with a fitness function based on measured runtime performance, 
to search for the best performing chromosomes.

The \postopt{} phase eliminates bangs from the best-performing
chromosomes returned from \Ao{}.  It does so by first turning off all
bangs that lie within a \coldspot{} before re-running the program
once for each gene that both lies in a \hotspot{} and is turned on in
the winning chromosomes. If a bang's absence negatively impacts performance
by at least the \absim{} threshold, it is kept in future rounds of testing and
will remain in the final combination of bangs for the optimized
program. If a bang does not meet the \absim{} threshold, it is removed
for future rounds of testing and will not appear in the final
combination. Similar to \Ao{}, the \postopt{} phase uses
the \textit{haskell-src-exts} library to parse the relevant source
files, set all the bangs appropriately, and then pretty-print the
modified source code to then be compiled by GHC and run on the
representative input.

Note that this process does not require re-running the original
profile.  Instead, we reuse the profiling information calculated
during the \preopt{} phase.  Bangs in \hotspots{} are tested in order
of decreasing costs. While we recognize that the order in which we
test the bangs may affect the performance, it is too time consuming to
test the bangs in all possible orders.  For simplicity, we chose to
test each bang once in order of decreasing costs.

Finally, the \postopt{} phase returns the final combination of bangs
that have survived each round of testing. If \Ao{} failed to find
chromosomes that improved program runtime by at least the
\overallThreshold{} parameter using genetic algorithms, then
the \postopt{} phase will return the unoptimized program.
If \At{} failed to find minimized bang placements that preserve 
program runtime, the \postopt{} phase will also 
return the unoptimized program. We base this choice on the
theory that the relatively insignificant performance improvement
indicates users are better off keeping the original program rather
than having to reason about the safety of the inferred bangs.
Currently, we set the both the \overallThreshold{} and \absim{}
value to 6\%, but users may change it. 

%----------------------------------------------------------------------------------------
%	SECTION 2
%----------------------------------------------------------------------------------------

\section{Running \At{}}

Users run \At{} the same way as they would run \Ao{}. They
provide a copy of their program source code, representative input,
and an optional configuration file. The \preopt{}
and \postopt{} phases typically require a negligible amount of time
to run compared with the time required to run \Ao{}.

{\renewcommand{\arraystretch}{1.5}
\begin{table}
\begin{tabular}{p{3cm}p{2cm}p{8cm}}
\hline
Term    &   Type    &   Description \\
\hline
\textit{diversityRate}  &   float   &   probability with which each gene 
                                        in seed is mutated to form initial 
                                        population\\
\textit{numGenerations} &   int     &   number of generations to run algorithm\\
\textit{populationSize} &   int     &   number of chromosomes in each population\\
\textit{archiveSize}    &   int     &   number of chromosomes to promote to
                                        next generation unchanged\\
\textit{mutateRate}     &   float   &   percentage of the new population 
                                        generated by mutation\\
\textit{mutateProb}     &   float   &   probability with which each gene in
                                        a chromosome selected for mutation is
                                        changed\\ 
\textit{crossRate}      &   float   &   percentage of the new population 
                                        generated by crossover\\
\textit{\absim{}}       &   float   &   threshold for evaluating whether the absence
                                        of a bang has a runtime impact significiant
                                        enough to preserve the bang \\
\textit{\hotspotcost{}} &   float   &   threshold for evaluating which cost center
                                        qualifies as a \hotspot{} based on its cost\\
\textit{profMetric}     &   string  &   metric to parse GHC profiles and 
                                        evaluate \hotspots{} with. (can be either
                                        "MEM" for memory allocations or "RT" for
                                        runtime)\\
\hline
\end{tabular}
\scaption{\At{} parameters}
\label{tab:A2params}
\end{table}}

\subsection{Choosing Configuration Parameters}

As shown in \figref{tab:A2params}, \At{} requires a few additional
configuration parameters than what \Ao{} already requires.
Choosing parameters that maximize performance optimization
and minimize bangs is difficult and the ideal threshold
will often vary from program to program. The \hotspotcost{} decides
whether a cost center is costly enough to be considered a \hotspot{}.
Higher \hotspotcost{} thresholds will result in fewer \hotspots{} and therefore
fewer bangs, at the risk of lower performance improvement due to too
many cost centers being eliminated. The \absim{} determines whether
a bang's contribution to performance impact is large enough to be preserved.
Higher \absim{} thresholds will similarly result in fewer preserved bangs
and therefore less manual inspection time, at the risk of lower performance
improvement due to too many bangs being eliminated. The \textit{profMetric} specifies
which metric to evaluate cost centers on, either using run time or memory
allocations. In our experiments, we found that depending on the program,
one \profm{} could be ideal over the other or not make a measurable
difference at all.


\subsection{\At{} Output}

If \At{} completes successfully, users can find the resulting source
code with the minimized bang annotations in the same project
directory as they would find the usual \Ao{} \texttt{survivor}
and \texttt{results} directories. If \preopt{} profiling detected that
the program is unsuitable for optimization, or if \Ao{} failed to
significantly optimize the program, then \At{} warns the user and
halts the optimization process. If external libraries could be added to
the optimization coverage to potentially improve the results,
the \preopt{} phase alerts the user and then continues to optimize as
normal.

%----------------------------------------------------------------------------------------
%   SECTION 3
%----------------------------------------------------------------------------------------

\section{Source Locations}

\Ao{} uses a bit vector to represent the genes in a
chromosome, with each bit corresponding to a possible bang location in
a source file.  In \At{}, we modified this representation so that each
bit also indicates the cost center associated with the possible bang
location.   Cost centers are uniquely identified by the corresponding source
locations in source files.  Thus, we mapped each bit to its corresponding
source line. To turn the bangs in a \hotspot{} on or off, we can
traverse the bit-location vector and manipulate the bits that are
tagged with source lines that fall within the range of
that \hotspot{}'s source location.

%----------------------------------------------------------------------------------------
%   SECTION 4
%----------------------------------------------------------------------------------------

\section{Removing Illegal Genes}

Both \Ao{} and \At{} use the \textit{haskell-src-exts} parser to add
and remove bangs.
Unfortunately, the version of \textit{haskell-src-exts} parser that we use
incorrectly identifies the left-hand side of bindings within instance
declarations as potential locations to place bangs. For that reason,
\Ao{} did not optimize files that contain instance declarations.
To eliminate this restriction, we instead removed any randomly
generated illegal bangs prior to each round of fitness evaluation in
the genetic algorithm. The rest of the genetic search runs
identically as before.

To keep track of whether a bang is legal, we used a
validity-indicating boolean vector to represent whether each gene in
the source code is legal. Prior to inserting bangs into a
program, \At{} would check the validity of a gene against the boolean
vector to make sure that the bang is located in a legal location.

Generically traversing the parser-generated AST using boilerplate code
fails to identify illegal genes, so we needed to manually traverse the AST
to construct the validity vector. As we traversed the AST, we kept
track of whether a left-hand side binding is within an instance
declaration. If so, then that binding is an illegal bang location and
is marked as \texttt{false} in the validity vector. All other
legal bang locations are marked as  \texttt{true}.

With this approach, \At{} successfully avoids inserting bangs into
illegal locations.  As a result, \At{} can now optimize files that
include instance declarations, which was not previously possible.
