\chapter{Related Work and Future Work} 

\label{Chapter6} % \ref{ChapterX}

%----------------------------------------------------------------------------------------
%	SECTION 1
%----------------------------------------------------------------------------------------

\section{\Ao{} and Other Methods of Managing Laziness}

The current strictness analyzer in GHC uses backward abstract
interpretation to identify locations that can be eagerly
evaluated~\cite{Sergey14}. The analysis is sound but necessarily
approximate because it is static and the underlying property is
undecidable. The analysis only marks locations as strict if it can
guarantee termination on all possible inputs, not just realistic ones,
since it is part of the compiler.  Like \Ao{}, \At{} leverages the
advantages of being dynamic and not attempting to guarantee
termination on all inputs.  Instead, \Ao{} allows users to decide if
the annotations are safe on the intended inputs.

Other approaches for reducing laziness include Strict
Haskell~\cite{strict-haskell}, which allows users to make entire modules strict rather than
lazy by default using the \texttt{-XStrict} and \texttt{-XStrictData} language
pragmas. Chang and Felleisen~\cite{Chang14} take the opposite
approach: they start with a program written in a strict
language and inserts laziness annotations using dynamic
profiling. It would be interesting to see if Chang and Felleisen's
method could be applied to introduce laziness to Strict Haskell
programs.

%----------------------------------------------------------------------------------------
%	SECTION 2
%----------------------------------------------------------------------------------------

\section{\At{} Improvements}

For future developments, it would be worth exploring methods to 
automatically predict which profiling metric would produce better
results so the user does not need to decide.
Similarly, we could implement thresholds
that automatically adjust themselves based on profiling results. Our
experiments show that the ideal values for \hotspotcost{} and \absim{}
thresholds vary by program to program. Adopting more flexible
thresholds that automatically adjust themselves after inspecting the
profile in the \preopt{} phase might yield better results than using
set values or asking users to provide them.

