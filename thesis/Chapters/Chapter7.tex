\chapter{Conclusion} 

\label{Chapter7} % \ref{ChapterX}

%----------------------------------------------------------------------------------------
%	SECTION 1
%----------------------------------------------------------------------------------------

\section{Conclusion}

Laziness is a double-edged sword: While it provides many benefits,
excessive laziness can cause poor performance. Strictness annotations
allow programmers to force eager evaluation, but its use is limited to
experienced programmers with high levels of expertise. \Ao{} uses a
genetic algorithm to automatically infer annotations to improve
program performance, but it often suggests too many bangs for users to
inspect. We have built \At{}, which uses GHC profiling feedback to
focus the search space and eliminates \useless{} bangs in a
post-processing phase.  Experiments show
that \At{} removes 90.2\% of bangs that \Ao{} recommended with only a
15.7\% optimization degradation on the NoFib benchmark,
and 81.8\% of the bangs with the same 15.7\% optimization degradation
on the \texttt{gcSimulator} case study.

